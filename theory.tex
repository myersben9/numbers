\documentclass{article}
\usepackage{amsmath}  % For mathematical symbols and equations
\usepackage{amssymb}  % For additional math symbols
\usepackage{graphicx} % For including images
\usepackage{hyperref} % For hyperlinks

\title{Notes on Quantitative Finance}
\author{Ben Myers}
\date{\today}

\begin{document}

\maketitle 

\section{Introduction}

In this section, we review fundamental mathematical concepts relevant to quantitative finance.

\subsection{Difference of Squares}
The difference of squares formula states:
\begin{equation}
    (a+b)(a-b) = a^2 - b^2.
\end{equation}
This identity is useful in various proofs and factorizations.

\subsection{Example: Proof that 899 is Not a Prime Number}

A prime number is a natural number greater than 1 that is only divisible by 1 and itself. To determine whether 899 is prime, we check if it has factors other than 1 and 899.

\paragraph{Step 1: Choosing Close Factors}  
We begin by approximating the square root of 899:
\[
\sqrt{899} \approx 30.
\]
We express 899 in the form of a difference of squares:
\[
899 = 30^2 - 1^2.
\]
Applying the identity:
\[
899 = (30 - 1)(30 + 1) = 29 \times 31.
\]

\paragraph{Step 2: Conclusion}  
Since 899 can be expressed as a product of two integers greater than 1, it is not a prime number.

\end{document}
